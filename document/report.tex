\documentclass{article}
\usepackage{amsmath}
\usepackage{bigints}
\begin{document}
1. Conservation of the total relative concentration $\Phi$ (Our order parameter)

Just like the total number of particles, $\Phi$ is conserved.
To demonstrate so, we start by reminding ourselves that $\phi(\mathbf{x},t) = c_A(\mathbf{x},t)-c_B(\mathbf{x},t)$.
Whence \[ \Phi(t) = \int_{\Omega} \phi(\mathbf{x},t)d\mathbf{x} = N_A - N_B, \]
where $\Omega$ is the available space.

$\Phi$ is thus conserved over time, since $N_A$ and $N_B$ also are.
As a side-note: \[ N = N_A + N_B = \int_{\Omega} d\mathbf{x} = \mu(\Omega), \]
which is the measure of the area of $\Omega$, which has 0 error for a fixed-size grid.

Since $2N_A = \Phi+1$, $\Phi$ is conserved if and only if $N_A$ and $N_B$ also are.

Finally, we define as an error metric the change in this quantity.
Numerically, we just calculate the sum of the relative concentration over the whole spatial grid.
We define
\[ \mathcal{E}(t) = |\Phi(t) - \Phi_0| \]
We take the logarithm since the scales are very small and note that for our tests $log_{2}\mathcal{E}(t) \approx -58$.
Which is on the order of the precision of the \verb`float64` type we used ($2^{-53}$).

About the function $f(\phi) = F(\phi) + \frac{\epsilon^2}{2}|\nabla{\phi}|^2$ (density of free energy):

The second term on $\phi$ describes the "gradient energy", or how the variability in the concentration locally increases the free energy.
So as to decrease the free energy, the system normally increases the entropy through mixing, this is accounted for in the first term.
But this second term is responsible for local regularity - locally, it favors regions of less variant concentration.
That is, it accounts for the decrease in the free energy due to the homogeneity in the system's composition.
(Higher homogeneity = Smaller local variation in composition = Smaller magnitude of concentration gradient = lower free energy).

In our simple model, this term accounts for the spinodal decomposition we observe in fluids.
For higher $\epsilon$ values, to minimize the Helmholtz free energy, we must make the gradient smaller overall, creating less but larger ``islands of constancy" pure in each phase.
This can also be seen as less ``regions of transition" between the 2 main phases.
So, $\epsilon$ can be seen as a term responsible for overall homogeneiry, which causes a tendency towards a smaller total surface of variation between the 2 phases.
So after long time steps we would expect spherical boundaries between the 2 main phases to form.
For relatively high $\epsilon$ this gives rise to very weak phase separation and thus results in a homogenous, miscible, monophasic fluid.
\end{document}
